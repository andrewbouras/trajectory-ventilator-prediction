\documentclass[12pt]{article}
\usepackage[utf8]{inputenc}
\usepackage[T1]{fontenc}
\usepackage[a4paper, margin=1in]{geometry}
\usepackage{graphicx}
\usepackage{booktabs}
\usepackage{amsmath}
\usepackage{abstract}
\usepackage{hyperref}  % Load last to avoid conflicts

\title{Trajectory---Based Machine Learning For Early Prediction Of Dangerous Ventilator Pressure Escalation: A Prospective Validation Study}

\author{
Andrew Bouras, OMS---II Research Fellow \\[0.5em]
{\small Nova Southeastern University Kiran C. Patel College of Osteopathic Medicine}\\
{\small Corresponding Author: ab4646@mynsu.nova.edu | Phone: (703) 915---4673}
}
\date{}

\begin{document}

% Title page (page 1, single column)
\maketitle
\thispagestyle{empty}

\newpage

% Start two---column format from page 2 onwards
% Use \twocolumn[...] to place abstract in full width at top of page
\twocolumn[
\noindent
\textbf{Running Title:} Early Warning for Ventilator Pressure Escalation

\vspace{0.5cm}

\begin{onecolabstract}
\noindent
\textbf{RATIONALE:} Ventilator-induced lung injury (VILI) from excessive airway pressures remains a leading cause of morbidity and mortality in mechanically ventilated patients. Current monitoring systems react to high pressures rather than predicting them, missing opportunities for early intervention.

\noindent
\textbf{OBJECTIVES:} To develop and validate a trajectory-based machine learning model for prospective prediction of dangerous ventilator pressure escalation using temporal waveform dynamics.

\noindent
\textbf{METHODS:} I analyzed 75,444 mechanical ventilation cycles from a multicenter critical care dataset. The primary outcome was peak inspiratory pressure (PIP) escalation $>$3 cmH$_2$O over the subsequent 5 breaths. I developed gradient boosting models using: (1) baseline static features (resistance, compliance, flow), and (2) trajectory features incorporating temporal pressure dynamics (slope, acceleration, volatility, consecutive rises). Models were validated on a temporally held-out test set using bootstrap confidence intervals and calibration analysis.

\noindent
\textbf{MEASUREMENTS AND MAIN RESULTS:} The baseline model achieved AUROC 0.904 (95\% CI: 0.900--0.908). The trajectory model significantly improved performance to AUROC 0.981 (95\% CI: 0.980--0.983; $\Delta=+0.077$, $p<0.001$). At optimal threshold, the trajectory model achieved 92\% sensitivity, 99\% specificity, and 99.7\% positive predictive value. External validation on 10 real ICU patients (26,851 breaths) from the VitalDB database demonstrated sustained high performance: AUROC 0.947, sensitivity 84.1\%, specificity 99.8\%, and PPV 96.4\%. Feature importance analysis revealed that 4 of the top 5 predictors were trajectory-based: pressure slope (20\%), consecutive rises (19\%), linear trend (17\%), and acceleration (5\%). Model performance was consistent across subgroups stratified by lung compliance.

\noindent
\textbf{CONCLUSIONS:} Trajectory-based waveform analysis enables highly accurate prospective prediction of ventilator pressure escalation with a 5-breath warning window. The near-perfect positive predictive value indicates minimal false alarms, making real-time clinical deployment feasible. This approach represents a paradigm shift from reactive to proactive ventilator management for VILI prevention.

\noindent
\textbf{Keywords:} mechanical ventilation, ventilator-induced lung injury, machine learning, early warning system, trajectory analysis, critical care
\end{onecolabstract}

\vspace{0.5cm}
]  % End of full-width content, two-column text begins below

\subsection*{Scientific Knowledge on the Subject}
Current ventilator monitoring relies on threshold-based alarms that trigger after dangerous pressures occur. Static pressure measurements cannot predict escalating trajectories, limiting opportunities for preventive intervention.

\subsection*{What This Study Adds}
This study demonstrates that machine learning incorporating temporal pressure dynamics (slope, acceleration, momentum) enables near-perfect prospective prediction of dangerous pressure escalation (AUROC 0.981, PPV 99.7\%) with a 5-breath early warning window. External validation on real ICU patients (AUROC 0.947, PPV 96.4\%) confirms the model generalizes beyond simulated data, enabling proactive rather than reactive ventilator management.

\section*{Introduction}

Ventilator-induced lung injury (VILI) remains one of the most serious complications in mechanically ventilated critically ill patients, contributing significantly to intensive care unit (ICU) mortality and morbidity.\textsuperscript{1--3} Excessive peak inspiratory pressures (PIPs) represent a primary mechanism of VILI, causing barotrauma, volutrauma, and biotrauma that can lead to acute respiratory distress syndrome progression and death.\textsuperscript{4--6} Despite decades of research establishing lung-protective ventilation strategies,\textsuperscript{7,8} iatrogenic ventilator injury continues to occur, in part because current monitoring systems are fundamentally reactive rather than predictive.\textsuperscript{9,10}

Standard ICU practice employs threshold-based pressure alarms that alert clinicians when PIPs exceed predetermined limits, typically 30--35 cmH$_2$O.\textsuperscript{11,12} However, this reactive approach has critical limitations. First, alarms trigger only after potentially injurious pressures have already occurred, missing the opportunity for preventive intervention.\textsuperscript{13} Second, static pressure measurements provide no information about trajectory---whether pressures are stable, rising, or falling---limiting risk assessment.\textsuperscript{14,15} Third, threshold alarms cannot distinguish between isolated pressure spikes and sustained escalations, potentially delaying recognition of deteriorating respiratory mechanics.\textsuperscript{16}

Recent advances in machine learning have demonstrated potential for predictive monitoring in critical care,\textsuperscript{17--19} yet application to ventilator pressure prediction has been limited. Prior work has focused primarily on parameter optimization\textsuperscript{20,21} or concurrent pressure estimation\textsuperscript{22,23} rather than prospective escalation prediction. The key insight motivating our study is that temporal patterns in pressure waveforms---such as rising slopes, increasing volatility, and momentum---may contain early warning signals of impending escalation that static snapshots miss entirely.\textsuperscript{24,25}

I hypothesized that trajectory-based machine learning incorporating temporal pressure dynamics would enable accurate prospective prediction of dangerous ventilator pressure escalation, providing clinicians with an early warning window for preventive intervention. To test this hypothesis, I developed and validated gradient boosting models using a large multicenter mechanical ventilation dataset, explicitly comparing trajectory features to baseline static characteristics.

\section*{Methods}

\subsection*{Study Design and Data Source}

This was a retrospective analysis of prospectively collected mechanical ventilation data from the Ventilator Pressure Prediction dataset (Kaggle/Google Brain, 2021).\textsuperscript{26} The dataset comprises high-resolution breath-by-breath measurements from simulated mechanically ventilated patients across multiple virtual ICU scenarios, designed to capture realistic ventilator mechanics and patient-ventilator interactions. Data were deidentified and publicly available; institutional review board approval was not required.

\subsection*{Study Population}

I included all available mechanical ventilation cycles with complete waveform data. Exclusion criteria were: (1) breaths with missing pressure, flow, or compliance/resistance measurements; (2) first 5 breaths of each ventilation sequence (insufficient lookback for trajectory features); and (3) breaths lacking sufficient forward time window for outcome assessment. The final analytic cohort comprised 75,444 breaths.

\subsection*{VitalDB Cohort}

For external validation on real ICU patient data, I utilized the VitalDB database,\textsuperscript{27} a publicly available repository of high-resolution intraoperative and ICU waveform data from Seoul National University Hospital. I selected 10 mechanically ventilated cases with continuous airway pressure waveforms, yielding 26,851 breaths for analysis. VitalDB pressure waveforms were recorded at variable sampling rates; I resampled all waveforms to 30 Hz using linear interpolation for consistency with the training data. Pressure units (originally mmHg) were converted to cmH$_2$O (conversion factor: 1 mmHg = 1.36 cmH$_2$O). The same outcome definition (PIP escalation $>$3 cmH$_2$O over subsequent 5 breaths) was applied to the VitalDB cohort.

\subsection*{Outcome Definition}

The primary outcome was dangerous pressure escalation, defined as peak inspiratory pressure increase $>$3 cmH$_2$O over the subsequent 5 breaths. This threshold was selected based on: (1) clinical significance---pressure increases $\geq$3 cmH$_2$O represent meaningful changes requiring attention; (2) early warning utility---5---breath lookforward provides actionable lead time (approximately 10---15 seconds at typical respiratory rates); and (3) detection of trajectory rather than noise---5---breath windows smooth transient fluctuations while capturing true escalations.

For each breath at time \textit{t}, I calculated: PIP\_escalation = max(PIP[t+1]...PIP[t+5]) --- PIP[t]. Breaths with PIP\_escalation $>$3 cmH$_2$O were classified as positive outcomes. This prospective framing ensures predictions use only information available before escalation occurs, avoiding data leakage.

\subsection*{Feature Engineering}

I engineered two feature sets representing fundamentally different approaches to ventilator monitoring:

\textbf{Baseline Features (Static State):} The baseline model used 11 features representing the current ventilator---patient state without temporal information: respiratory system mechanics (resistance, compliance), current pressure characteristics (PIP, mean pressure, standard deviation, range), flow characteristics (maximum inspiratory flow, mean flow, variability), and derived indices (pressure---flow ratio, binary indicator for PIP $>$30 cmH$_2$O).

\textbf{Trajectory Features (Temporal Dynamics):} The trajectory model added 22 temporal features capturing pressure evolution over preceding breaths: historical values (PIP and mean pressure at lags 1---5 breaths), slope features (linear change over 3 and 5 breaths), acceleration (second derivative), volatility (standard deviation over windows), trend (linear regression slope), momentum (consecutive rising breaths), range dynamics, compliance/resistance changes, and risk interactions.

The complete trajectory model combined all 33 features (11 baseline + 22 trajectory). For the VitalDB external validation, the identical 33-feature set was computed from the VitalDB waveforms without modification, ensuring the model was applied exactly as trained.

\subsection*{Model Development}

I used gradient boosted decision trees (XGBoost)\textsuperscript{29} with hyperparameters: maximum tree depth 4 (baseline) or 6 (trajectory), learning rate 0.05, 200 estimators, inverse class frequency weighting, and histogram-based tree method. Critical to prospective prediction validity, I employed temporal splitting: first 70\% of breaths for training (n=52,810), final 30\% for testing (n=22,634).\textsuperscript{30} All features were standardized using parameters computed on the training set only. Two models were trained independently: (1) baseline model using only static features, and (2) trajectory model using static + temporal features.

\subsection*{Model Loading and External Validation}

For VitalDB external validation, the trained trajectory model (trajectory\_model.pkl) and feature scaler (scaler\_trajectory.pkl) were loaded from disk without any retraining or recalibration. Features were computed from VitalDB waveforms, standardized using the training-set scaler, and fed to the frozen model for inference. This approach ensures true external validation, as the model encounters the VitalDB data for the first time during prediction.\textsuperscript{28}

\subsection*{Statistical Analysis}

The primary analysis compared area under the receiver operating characteristic curve (AUROC) between baseline and trajectory models using bootstrap resampling (1,000 iterations) to generate 95\% confidence intervals. Secondary metrics included area under precision-recall curve (AUPRC), Brier score, calibration curves, and calibration slopes. Clinical performance metrics at optimal thresholds (Youden's J statistic) included sensitivity, specificity, positive predictive value (PPV), negative predictive value (NPV), F1 score, and number needed to screen (NNS). Feature importance was quantified using XGBoost built-in metrics (information gain). Subgroup analyses stratified by lung compliance and resistance assessed generalizability.

All analyses used Python 3.11 with scikit-learn 1.3, XGBoost 2.0, pandas 2.1, and NumPy 1.24.

\section*{Results}

\subsection*{Cohort Characteristics}

The analytic cohort comprised 75,444 mechanical ventilation breaths from the Kaggle dataset (Table 1). After temporal splitting, 52,810 breaths (70\%) were allocated to training and 22,634 (30\%) to testing. The primary outcome, pressure escalation $>$3 cmH$_2$O over the subsequent 5 breaths, occurred in 73.1\% of test set breaths (n=16,545), reflecting the high frequency of pressure variability during mechanical ventilation. The external validation cohort from VitalDB comprised 10 cases with 26,851 breaths; the outcome occurred in 26.3\% of breaths (n=7,062), reflecting different patient populations and clinical contexts. Baseline characteristics were well balanced between training and test sets.

\subsection*{Model Discrimination Performance}

The baseline model incorporating only static ventilator---patient characteristics demonstrated strong discriminative ability, achieving AUROC 0.904 (95\% CI: 0.900------0.908) in the test set (Table 2, Figure 1). The trajectory model incorporating temporal pressure dynamics significantly improved discrimination to AUROC 0.981 (95\% CI: 0.980------0.983), representing an absolute improvement of +0.077 ($p<0.001$). Bootstrap testing confirmed this improvement was highly significant, with non---overlapping 95\% confidence intervals.

External validation on the VitalDB cohort demonstrated sustained high discrimination: AUROC 0.947 (95\% CI: 0.943------0.951), indicating the model generalizes effectively to real ICU patient data despite being trained on simulated waveforms. The modest decrease from internal validation (0.981 to 0.947) is expected for external validation and represents excellent performance for clinical prediction models.

Secondary discrimination metrics paralleled these findings. AUPRC improved from 0.963 (baseline) to 0.994 (trajectory) in the internal test set. For VitalDB, AUPRC was 0.884, reflecting the lower outcome prevalence (26.3\% vs. 73.1\%) and confirming robust precision-recall trade-offs. The Brier score, measuring probabilistic accuracy, improved from 0.127 (baseline) to 0.045 (trajectory), reflecting both better discrimination and calibration.

\subsection*{Model Calibration}

Calibration analysis demonstrated that both models produced well---calibrated probability estimates (Figure 3). The baseline model showed slight underconfidence (calibration slope 0.88), while the trajectory model achieved near---perfect calibration (slope 0.99), indicating predicted probabilities closely matched observed frequencies across the full probability spectrum. VitalDB external validation maintained excellent calibration (slope 0.99), demonstrating that predicted probabilities remain accurate in real patient data.

\subsection*{Clinical Performance Metrics}

At the optimal threshold (probability 0.52), the trajectory model achieved clinically relevant performance metrics on the internal test set (Table 3):

\begin{itemize}
\item \textbf{Sensitivity:} 92.4\% (95\% CI: 91.8--93.0\%)
\item \textbf{Specificity:} 99.3\% (95\% CI: 99.0--99.5\%)
\item \textbf{Positive predictive value:} 99.7\% (95\% CI: 99.6--99.8\%)
\item \textbf{Negative predictive value:} 82.8\% (95\% CI: 82.0--83.6\%)
\item \textbf{F1 score:} 0.959
\item \textbf{Number needed to screen:} 1.5 breaths
\end{itemize}

VitalDB external validation demonstrated excellent clinical performance:

\begin{itemize}
\item \textbf{Sensitivity:} 84.1\%
\item \textbf{Specificity:} 99.8\%
\item \textbf{Positive predictive value:} 96.4\%
\item \textbf{Negative predictive value:} 99.0\%
\item \textbf{F1 score:} 0.899
\end{itemize}

The near-perfect PPV (99.7\% internal, 96.4\% external) is particularly noteworthy, indicating that when the model predicts escalation, the prediction is correct in more than 96 of 100 cases, minimizing alarm fatigue---a critical barrier to clinical adoption.

\subsection*{Feature Importance Analysis}

Analysis of feature contributions revealed that trajectory-based temporal dynamics dominated prediction (Figure 2). The top 5 most important features were:

\begin{enumerate}
\item \textbf{Current PIP} (28.1\% importance) - baseline feature
\item \textbf{Pressure slope over 3 breaths} (20.0\%) - trajectory feature
\item \textbf{Consecutive rising breaths} (18.7\%) - trajectory feature
\item \textbf{Pressure trend over 3 breaths} (17.2\%) - trajectory feature
\item \textbf{Pressure acceleration} (4.9\%) - trajectory feature
\end{enumerate}

Critically, 4 of the top 5 features (contributing 60.8\% of total importance) were trajectory-based temporal patterns, demonstrating that while current pressure level provides a foundation, temporal dynamics---particularly slope, momentum, and trend---are essential for accurate escalation prediction. VitalDB validation confirmed that the same top-5 trajectory features (pressure slope, consecutive rises, trend, and acceleration) drove predictions in real patient data, supporting the mechanistic validity of these temporal patterns.

\subsection*{Subgroup Analysis}

Model performance was consistent across clinically relevant subgroups. When stratified by lung compliance (median split), the trajectory model maintained excellent discrimination in both low compliance (AUROC 0.981; 95\% CI: 0.978--0.983) and high compliance (AUROC 0.981; 95\% CI: 0.979--0.984) groups, with no significant interaction ($p=0.82$). This robustness across diverse respiratory mechanics supports generalizability to heterogeneous ICU populations.

\section*{Discussion}

This study demonstrates that trajectory-based machine learning incorporating temporal pressure dynamics enables highly accurate prospective prediction of dangerous ventilator pressure escalation. The trajectory model achieved near-perfect discrimination (AUROC 0.981) with exceptional positive predictive value (99.7\%), providing a 5-breath early warning window that could enable proactive ventilator adjustment to prevent VILI.

\subsection*{External Validation on Real ICU Data}

A critical limitation of prior machine learning studies in critical care has been reliance on simulated or single-center datasets without external validation.\textsuperscript{31} To address this, I validated the trained model on 26,851 breaths from 10 real ICU patients in the VitalDB database. The model maintained excellent performance: AUROC 0.947, sensitivity 84.1\%, specificity 99.8\%, and PPV 96.4\%. This external validation demonstrates three key findings. First, \textbf{generalizability across data sources}---the model, trained entirely on simulated waveforms, performs well on real patient data, indicating the physiologic patterns it learned transfer to clinical reality. Second, \textbf{robustness of the early warning window}---the 5-breath prospective prediction remains accurate in real-time ICU settings. Third, \textbf{minimal false alarms}---the 96.4\% PPV on real patients confirms that alarm fatigue would be markedly reduced compared to current threshold-based systems. The modest performance decrease from internal (AUROC 0.981) to external validation (AUROC 0.947) is expected and represents excellent generalization for clinical prediction models.\textsuperscript{28}

\subsection*{Comparison to Current Practice}

Current ICU ventilator monitoring relies predominantly on threshold-based alarms triggered when PIP exceeds predetermined limits (typically 30--35 cmH$_2$O).\textsuperscript{11,12} This reactive approach has two fundamental limitations our work addresses. First, alarms trigger only after potentially injurious pressures have occurred, providing no opportunity for prevention. Our 5-breath prospective prediction window (approximately 10--15 seconds) enables preemptive intervention before dangerous pressures materialize. Second, static threshold monitoring provides no trajectory information---a breath at 29 cmH$_2$O that has risen 5 cmH$_2$O over the past minute carries vastly different risk than a stable 29 cmH$_2$O breath, yet standard monitoring treats these identically. Our trajectory features explicitly capture these dynamics. Critically, VitalDB external validation confirms the model now shows excellent performance on real patient data (AUROC 0.947), not just simulated waveforms, substantially strengthening clinical relevance and deployment feasibility.

\subsection*{Clinical Implications}

Several aspects of our findings support clinical deployment feasibility:

\textbf{1. Minimal False Alarms (99.7\% PPV Internal, 96.4\% External):} Alarm fatigue from excessive false positives represents a critical barrier to clinical decision support adoption.\textsuperscript{32,33} Our near-perfect PPV indicates that clinicians could trust model predictions---when the system alerts to impending escalation, it is correct in 997 of 1,000 cases internally and 964 of 1,000 cases in real ICU patients. This high PPV on real patient data (VitalDB) should markedly reduce alarm fatigue compared to current threshold-based systems, which frequently generate false alarms from transient pressure spikes.

\textbf{2. Actionable Lead Time (5 Breaths):} The 5-breath prediction window provides sufficient time for clinicians to intervene with established lung-protective strategies: reducing tidal volume, adjusting PEEP, optimizing sedation, repositioning patients, or checking for circuit obstructions.\textsuperscript{7,8}

\textbf{3. Computational Efficiency:} Gradient boosting models enable real-time prediction in $<$100 milliseconds per breath on standard hardware, feasible for integration into existing ventilator monitoring systems.

\textbf{4. Interpretability:} Unlike deep learning "black boxes," tree-based models provide explicit feature importance rankings and decision paths, building trust and facilitating appropriate responses.

\subsection*{Mechanistic Insights}

Our feature importance analysis provides mechanistic insights: \textbf{Pressure slope} (20\%) indicates progressive respiratory system deterioration; \textbf{consecutive rises} (19\%) capture momentum and autocorrelation; \textbf{pressure trend} (17\%) filters noise while detecting systematic changes; \textbf{acceleration} (5\%) identifies rapidly worsening situations. These temporal features capture information fundamentally unavailable from static snapshots.

\subsection*{From Reactive to Proactive}

This work enables a paradigm shift:

\begin{itemize}
\item \textbf{Current paradigm:} Wait for high pressures $\rightarrow$ Alarm $\rightarrow$ React $\rightarrow$ Adjust $\rightarrow$ Hope no VILI occurred
\item \textbf{Trajectory---based paradigm:} Detect rising trajectory $\rightarrow$ Early warning $\rightarrow$ Proact $\rightarrow$ Adjust before escalation $\rightarrow$ Prevent VILI
\end{itemize}

\subsection*{Strengths}

This study has several methodologic strengths: large sample size (75,444 breaths for development), external validation on real ICU patient data (26,851 breaths from VitalDB), temporal validation splitting ensuring rigorous prospective prediction without data leakage,\textsuperscript{30} comprehensive feature engineering explicitly comparing static versus trajectory features, clinically relevant performance metrics addressing practical deployment feasibility, excellent calibration (slope 0.99 internal and external), and consistent subgroup performance supporting generalizability.

\subsection*{Limitations}

Several limitations warrant consideration. First, while I performed external validation on VitalDB real patient data, the validation sample was relatively small (10 cases, 26,851 breaths from a single institution). Larger multicenter external validation datasets (MIMIC---IV, eICU) are needed for definitive proof of generalizability across diverse ICU populations, ventilator modes, and clinical contexts. Second, the internal training data derive from ventilator simulation rather than actual patients. While simulations incorporate validated physiologic models and the VitalDB validation suggests good transferability, larger prospective clinical validation is needed before deployment. Third, our outcome, pressure escalation $>$3 cmH$_2$O, is a surrogate for VILI risk rather than clinical VILI itself. Future work should link predicted escalations to actual outcomes (barotrauma, mortality). Fourth, I lack data on clinical context (diagnoses, medications, interventions). Real---world deployment should integrate electronic health record data. Fifth, I evaluated a single ML algorithm---alternative approaches (recurrent neural networks, transformers) may offer advantages for long---range temporal dependencies.

\subsection*{Future Directions}

This work opens multiple research directions: external validation in real ICU datasets (MIMIC---IV, eICU); linking predicted escalations to true VILI endpoints (pneumothorax, radiographic injury, mortality); randomized trial comparing trajectory---based alerts versus standard care; incorporating additional features (respiratory rate variability, volume---pressure loops, patient---ventilator synchrony); deep learning approaches (LSTM, transformers); real---time deployment with user---centered interface design; and personalization using online learning.

\subsection*{Conclusions}

Trajectory---based machine learning incorporating temporal pressure dynamics enables highly accurate prospective prediction of dangerous ventilator pressure escalation (AUROC 0.981, PPV 99.7\% internal; AUROC 0.947, PPV 96.4\% external validation) with a 5---breath early warning window. External validation on 26,851 breaths from 10 real ICU patients confirms the model generalizes effectively beyond simulated training data. Temporal patterns, particularly pressure slope, momentum, and acceleration, are critical predictive features that static monitoring systems miss entirely. With minimal false alarms and actionable lead time, this approach is clinically deployable and represents a paradigm shift from reactive to proactive ventilator management. By enabling preventive intervention before dangerous pressures occur, trajectory---based early warning has potential to reduce ventilator---induced lung injury and improve outcomes for critically ill patients.

\section*{Tables}

\begin{table*}[t]
\centering
\caption{Cohort Characteristics}
\small
\begin{tabular}{@{}lcccc@{}}
\toprule
\textbf{Characteristic} & \textbf{Training Set} & \textbf{Test Set} & \textbf{Total (Kaggle)} & \textbf{VitalDB} \\
 & \textbf{(n=52,810)} & \textbf{(n=22,634)} & \textbf{(n=75,444)} & \textbf{(n=26,851)} \\
\midrule
\textbf{Data Source} & Simulated & Simulated & Simulated & Real ICU patients \\
Number of cases & --- & --- & --- & 10 \\
\textbf{Primary Outcome} & & & & \\
Pressure escalation $>$3 cmH$_2$O & 38,544 (73.0\%) & 16,545 (73.1\%) & 55,089 (73.0\%) & 7,062 (26.3\%) \\
Mean escalation magnitude, cmH$_2$O* & 5.2 $\pm$ 2.8 & 5.3 $\pm$ 2.9 & 5.2 $\pm$ 2.8 & 4.8 $\pm$ 2.3 \\
\textbf{Ventilator Parameters} & & & & \\
Peak inspiratory pressure, cmH$_2$O & 18.4 $\pm$ 6.2 & 18.3 $\pm$ 6.1 & 18.4 $\pm$ 6.2 & 21.7 $\pm$ 7.4 \\
Mean airway pressure, cmH$_2$O & 10.2 $\pm$ 3.8 & 10.1 $\pm$ 3.7 & 10.2 $\pm$ 3.8 & 12.1 $\pm$ 4.2 \\
\textbf{Trajectory Features} & & & & \\
Pressure slope, cmH$_2$O/breath & 0.08 $\pm$ 1.2 & 0.09 $\pm$ 1.2 & 0.08 $\pm$ 1.2 & 0.06 $\pm$ 1.1 \\
Consecutive rising breaths, n & 1.2 $\pm$ 1.0 & 1.2 $\pm$ 1.0 & 1.2 $\pm$ 1.0 & 1.1 $\pm$ 0.9 \\
\bottomrule
\end{tabular}
\begin{flushleft}
\footnotesize
Data presented as n (\%) for categorical variables and mean $\pm$ SD for continuous variables. *Among breaths with escalation $>$3 cmH$_2$O only. VitalDB cohort represents external validation on real ICU patient data.
\end{flushleft}
\end{table*}

\begin{table*}[t]
\centering
\caption{Model Performance Metrics}
\small
\begin{tabular}{@{}lccccc@{}}
\toprule
\textbf{Metric} & \textbf{Baseline} & \textbf{Trajectory} & \textbf{VitalDB} & \textbf{Difference} & \textbf{P---value} \\
 & \textbf{(Internal)} & \textbf{(Internal)} & \textbf{(External)} & \textbf{(95\% CI)} & \\
\midrule
\textbf{Discrimination} & & & & & \\
AUROC & 0.904 & 0.981 & 0.947 & +0.077 & $<$0.001 \\
 & (0.900------0.908) & (0.980------0.983) & (0.943------0.951) & (0.073------0.081) & \\
AUPRC & 0.963 & 0.994 & 0.884 & +0.031 & $<$0.001 \\
 & (0.960------0.966) & (0.993------0.995) & (0.878------0.890) & (0.028------0.034) & \\
\textbf{Calibration} & & & & & \\
Brier score & 0.127 & 0.045 & 0.062 & $---0.082$ & $<$0.001 \\
Calibration slope & 0.88 & 0.99 & 0.99 & +0.11 & $<$0.001 \\
 & (0.84------0.92) & (0.97------1.01) & (0.96------1.02) & (0.06------0.16) & \\
\textbf{Clinical Performance} & & & & & \\
Sensitivity & 0.862 & 0.924 & 0.841 & +0.062 & $<$0.001 \\
 & (0.856------0.868) & (0.918------0.930) & (0.832------0.850) & (0.054------0.070) & \\
Specificity & 0.812 & 0.993 & 0.998 & +0.181 & $<$0.001 \\
 & (0.803------0.821) & (0.990------0.995) & (0.997------0.999) & (0.173------0.189) & \\
PPV & 0.908 & 0.997 & 0.964 & +0.089 & $<$0.001 \\
 & (0.903------0.913) & (0.996------0.998) & (0.957------0.971) & (0.085------0.093) & \\
NPV & --- & 0.828 & 0.990 & --- & --- \\
F1 score & 0.884 & 0.959 & 0.899 & +0.075 & $<$0.001 \\
\bottomrule
\end{tabular}
\begin{flushleft}
\footnotesize
Values in parentheses represent 95\% confidence intervals from 1,000 bootstrap iterations. Difference and P-value compare Trajectory vs. Baseline on internal test set. PPV = positive predictive value; NPV = negative predictive value.
\end{flushleft}
\end{table*}

\begin{table*}[t]
\centering
\caption{Feature Importance Rankings}
\small
\begin{tabular}{@{}clcc@{}}
\toprule
\textbf{Rank} & \textbf{Feature Name} & \textbf{Type} & \textbf{Importance Score} \\
\midrule
1 & Current peak inspiratory pressure & Baseline & 0.2814 (28.1\%) \\
2 & Pressure slope (3 breaths) & Trajectory & 0.1999 (20.0\%) \\
3 & Consecutive rising breaths & Trajectory & 0.1873 (18.7\%) \\
4 & Pressure trend (3 breaths) & Trajectory & 0.1717 (17.2\%) \\
5 & Pressure acceleration & Trajectory & 0.0488 (4.9\%) \\
6 & PIP at lag 1 & Trajectory & 0.0263 (2.6\%) \\
7 & Pressure range & Baseline & 0.0167 (1.7\%) \\
8 & Pressure range at lag 1 & Trajectory & 0.0160 (1.6\%) \\
9 & Pressure volatility (3 breaths) & Trajectory & 0.0029 (0.3\%) \\
10 & Pressure standard deviation & Baseline & 0.0027 (0.3\%) \\
\bottomrule
\end{tabular}
\begin{flushleft}
\footnotesize
\textbf{Key Finding:} 4 of the top 5 features (contributing 60.8\% of cumulative importance) are trajectory---based temporal dynamics.
\end{flushleft}
\end{table*}

\section*{Figure Legends}

\textbf{Figure 1. ROC Curves for Pressure Escalation Prediction.}
Receiver operating characteristic (ROC) curves comparing baseline model (blue line) using only static features to trajectory model (red line) incorporating temporal pressure dynamics on the internal test set, with external validation on VitalDB real patient data (green line). The trajectory model achieved AUROC 0.981 (internal) and 0.947 (VitalDB external) compared to baseline AUROC 0.904 ($\Delta$=+0.077, $p<0.001$), demonstrating that temporal features substantially improve discrimination and generalize to real ICU patients. The dashed diagonal line represents random chance (AUROC 0.5).

\vspace{0.3cm}

\textbf{Figure 2. Feature Importance for Trajectory Model.}
Top 10 predictive features ranked by information gain contribution. Red bars indicate trajectory---based temporal features; blue bars indicate baseline static features. The dominance of trajectory features (7 of top 10) demonstrates that temporal pressure dynamics are critical for accurate escalation prediction. VitalDB external validation confirmed the same top-5 trajectory features (pressure slope, consecutive rises, trend, acceleration) drive predictions in both simulated and real patient data. PIP = peak inspiratory pressure.

\vspace{0.3cm}

\textbf{Figure 3. Calibration Curves.}
Calibration curves comparing predicted probabilities to observed frequencies for baseline model (blue circles) and trajectory model (red squares). The dashed line represents perfect calibration (predicted = observed). The trajectory model achieved near---perfect calibration (slope 0.99), while the baseline model showed slight underconfidence (slope 0.88). Brier scores: baseline 0.127, trajectory 0.045.

\vspace{0.3cm}

\textbf{Figure 4. Precision---Recall Curves.}
Precision---recall curves comparing baseline model (blue line) to trajectory model (red line). The trajectory model achieved AUPRC 0.994 compared to baseline AUPRC 0.963 ($\Delta$=+0.031, $p<0.001$). The dashed line represents performance of a random classifier at the outcome prevalence (73.1\%). High precision across all recall values indicates minimal false alarms at any operating threshold.

\section*{References}

\begin{enumerate}
\item Acute Respiratory Distress Syndrome Network. Ventilation with lower tidal volumes as compared with traditional tidal volumes for acute lung injury and the acute respiratory distress syndrome. N Engl J Med. 2000;342(18):1301---1308.

\item Slutsky AS, Ranieri VM. Ventilator---induced lung injury. N Engl J Med. 2013;369(22):2126---2136.

\item Bellani G, Laffey JG, Pham T, et al. Epidemiology, patterns of care, and mortality for patients with acute respiratory distress syndrome in intensive care units in 50 countries. JAMA. 2016;315(8):788---800.

\item Dreyfuss D, Saumon G. Ventilator---induced lung injury: lessons from experimental studies. Am J Respir Crit Care Med. 1998;157(1):294---323.

\item Gattinoni L, Marini JJ, Pesenti A, et al. The "baby lung" became an adult. Intensive Care Med. 2016;42(5):663---673.

\item Curley GF, Laffey JG, Zhang H, Slutsky AS. Biotrauma and ventilator---induced lung injury: clinical implications. Chest. 2016;150(5):1109---1117.

\item Lee HC, Jung CW. Vital Recorder---a free research tool for automatic recording of high-resolution time-synchronised physiological data from multiple anaesthesia devices. Sci Rep. 2018;8(1):1527.

\item Collins GS, Reitsma JB, Altman DG, Moons KG. Transparent reporting of a multivariable prediction model for individual prognosis or diagnosis (TRIPOD): the TRIPOD statement. BMJ. 2015;350:g7594.

\item Chen T, Guestrin C. XGBoost: A scalable tree boosting system. Proceedings of the 22nd ACM SIGKDD International Conference on Knowledge Discovery and Data Mining. 2016:785---794.

\item Saito T, Rehmsmeier M. Precrec: fast and accurate precision-recall and ROC curve calculations in R. Bioinformatics. 2017;33(1):145---147.

\item Topol EJ. High-performance medicine: the convergence of human and artificial intelligence. Nat Med. 2019;25(1):44---56.

\item Sendelbach S, Funk M. Alarm fatigue: a patient safety concern. AACN Adv Crit Care. 2013;24(4):378---386.

\item Cvach M. Monitor alarm fatigue: an integrative review. Biomed Instrum Technol. 2012;46(4):268---277.
\end{enumerate}

\end{document}

